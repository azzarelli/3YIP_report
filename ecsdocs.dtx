% \iffalse
% This file builds on the class file `book.cls' to provide a thesis,
% minithesis, progress, project and report format, and on the class file `article.cls'
% to provide an article format, for use within the Dept. of Electronics
% and Computer Science at the University of Southampton.
%
% The thesis, minithesis, progress, project, report, gdp, gdpsummary and article environments are written to separate
% class files ecsthesis.cls, ecsminithesis.cls, ecsprogress.cls, ecsproject.cls, ecsreport.cls, ecsgdp.cls, ecsgdpsummary.cls and ecsarticle.cls.
% \fi
%
% \CheckSum{1773}
%
% \iffalse
% Copyright (C) 2001 by Steve R. Gunn
%
%    \begin{macrocode}
%<thesis|minithesis|progress|project|article|report|gdp|gdpsummary>\NeedsTeXFormat{LaTeX2e}[1996/12/01]
%    \end{macrocode}
%
%    Announce the document class and its version.
%
%    \begin{macrocode}
%<thesis>\ProvidesClass{ecsthesis}
%<minithesis>\ProvidesClass{ecsminithesis}
%<progress>\ProvidesClass{ecsprogress}
%<project>\ProvidesClass{ecsproject}
%<article>\ProvidesClass{ecsarticle}
%<report>\ProvidesClass{ecsreport}
%<gdp>\ProvidesClass{ecsgdp}
%<gdpsummary>\ProvidesClass{ecsgdpsummary}
%<*driver>
\ProvidesFile{ecsdocs.drv}
%</driver>
%<*thesis|minithesis|progress|project|report|article|gdp|gdpsummary|driver>
              [2003/25/04 v1.6
%</thesis|minithesis|progress|project|report|article|gdp|gdpsummary|driver>
%<thesis|minithesis|progress|project|report|article|gdp|gdpsummary>   LaTeX document class]
%    \end{macrocode}
%
%    This bit of code contains the documentation driver file for
%    \TeX{}, i.e., the file that will produce the documentation you
%    are currently reading. It can be extracted from this file by the
%    {\sc docstrip} program.
%    \begin{macrocode}
%<*driver>
]
\documentclass[a4paper,10pt]{ltxdoc}
\GetFileInfo{ecsdocs.drv} \CodelineIndex \EnableCrossrefs
%\RecordChanges
\setcounter{IndexColumns}{2}
\begin{document}
\DocInput{ecsdocs.dtx}
\PrintIndex
%\PrintChanges
\end{document}
%</driver>
%    \end{macrocode}
% \fi
%
% \title{\LaTeXe{} Document Styles for the\\
%        Department of Electronics and Computer Science\\
%       {File version \fileversion, dated \filedate}}
%
% \author{Steve R. Gunn}
%
% \date{Printed \today}
%
% \maketitle
%
% \section{Description}
%
%    These \LaTeXe{} document classes are designed to produce thesis, mini-thesis, progress,
%    project, report, GDP report and GDP summary report documents which conform to the
%    University of Southampton guidelines of April 2001. They are based on the document
%    class \texttt{book}, but modify some of its layout decisions. An article class is also
%    provided which is based on the document class \texttt{article}.
%
% \StopEventually{}
%
% \section{The {\sc docstrip} modules}
%
% The following modules are used in the implementation to direct
% {\sc docstrip} when generating the external files:
% \begin{center}
% \begin{tabular}{ll}
%   thesis            & produce the \texttt{ecsthesis} document class\\
%   minithesis        & produce the \texttt{ecsminithesis} document class\\
%   progress          & produce the \texttt{ecsprogress} document class\\
%   project           & produce the \texttt{ecsproject} document class\\
%   report            & produce the \texttt{ecsreport} document class\\
%   article           & produce the \texttt{ecsarticle} document class\\
%   gdp               & produce the \texttt{ecsgdp} document class\\
%   gdpsummary        & produce the \texttt{ecsgdpsummary} document class\\
%   testthesis        & produce an example thesis\\
%   testminithesis    & produce an example minithesis\\
%   testprogress      & produce an example progress\\
%   testproject       & produce an example project\\
%   testreport        & produce an example report\\
%   testarticle       & produce an example article\\
%   testgdp           & produce an example GDP report\\
%   testgdpsummary    & produce an example GDP summary report\\
%   introduction      & produce an example introduction chapter\\
%   conclusions       & produce an example conclusions chapter\\
%   appendix          & produce an example appendix chapter\\
%   figure            & produce an example figure file\\
%   definitions       & produce an example definitions file\\
%   references        & produce an example references file\\
%   bst               & produce a bibliography style file\\
%   driver            & produce a documentation driver file \\
% \end{tabular}
% \end{center}
%
% \section{Implementation}
%
%    All options are passed on to the \texttt{book} or \texttt{article} class.
%
%    \begin{macrocode}
%<*thesis|minithesis|progress|project|report|article|gdp|gdpsummary>
%    \end{macrocode}
%
% \begin{macro}{\baseclass}
%    \begin{macrocode}
%<*thesis|minithesis|progress|project|report|gdp|gdpsummary>
\def\baseclass{book}
%</thesis|minithesis|progress|project|report|gdp|gdpsummary>
%<*article>
\def\baseclass{article}
%</article>
%    \end{macrocode}
% \end{macro}
%
%    \begin{macrocode}
\DeclareOption*{\PassOptionsToClass{\CurrentOption}{\baseclass}}
%    \end{macrocode}
%
% \begin{macro}{\@checkoptions}
%
%    Before we can process the options we must deal with a small
%    problem.  That is, we want to change the default options of
%    the \texttt{book} or \texttt{article} class to \texttt{a4paper} and \texttt{11pt} or \texttt{12pt} but the
%    macro |\ProcessOptions| used by the \texttt{book} or \texttt{article} class
%    processes the options in the order specified in the
%    file.  Because \texttt{a4paper} is the first of the paper
%    sizing options it will be overridden by any user defined
%    option.  But the same is not true of the font size option
%    \texttt{11pt} or \texttt{12pt}.  To remedy this we define a macro
%    |\@checkoptions{|\meta{default}|}{|\meta{option-list}|}|
%    which checks to see if any options in \meta{option-list} have
%    been specified and if not adds \meta{default} to the option list.
%
%    \begin{macrocode}
\def\@checkoptions#1#2{
  \edef\@curroptions{\@ptionlist{\@currname.\@currext}}
  \@tempswafalse
  \@tfor\@this:=#2\do{
    \@expandtwoargs\in@{,\@this,}{,\@curroptions,}
    \ifin@ \@tempswatrue \@break@tfor \fi}
  \let\@this\@empty
  \if@tempswa \else \PassOptionsToClass{#1}{\baseclass}\fi
}
%    \end{macrocode}
% \end{macro}
%
%    With this macro at our disposal we can continue to
%    process the options
%
%    \begin{macrocode}
%<*thesis|minithesis|progress|report|article|gdp|gdpsummary>
\@checkoptions{11pt}{{10pt}{11pt}{12pt}}
%</thesis|minithesis|progress|report|article|gdp|gdpsummary>
%<*project>
\@checkoptions{12pt}{{10pt}{11pt}{12pt}}
%</project>
\PassOptionsToClass{a4paper}{\baseclass}
\ProcessOptions\relax
%    \end{macrocode}
%
%    and load the \texttt{book} or \texttt{article} document class
%    with A4 as the default paper size.
%
%    \begin{macrocode}
\LoadClass{\baseclass}
%    \end{macrocode}
%
% \subsection{Banner}
%
% \begin{macro}{\bhrule}
% \begin{macro}{\btypeout}
%
%    Provide a function to produce a banner on the output making the location
%    of the current section simpler.
%
%    \begin{macrocode}
\newcommand\bhrule{\typeout{------------------------------------------------------------------------------}}
\newcommand\btypeout[1]{\bhrule\typeout{\space #1}\bhrule}
%    \end{macrocode}
% \end{macro}
% \end{macro}
%
% \subsection{Date}
%
% \begin{macro}{\today}
%
%    This macro uses the \TeX\ primitives |\month|, |\day| and |\year|
%    to provide the date of the \LaTeX-run in a suitable format.
%
%    \begin{macrocode}
%<*thesis>
\def\today{\ifcase\month\or
  January\or February\or March\or April\or May\or June\or
  July\or August\or September\or October\or November\or December\fi
  \space \number\year}
%</thesis>
%    \end{macrocode}
%
% \end{macro}
%
% \section{Document Layout}
%
% \subsection{Spacing}
%
%    Load the \texttt{setspace} package so that spacing can easily be set. The text in all
%    three styles is set to one and a half spacing, paragraphs not indented, and a newline
%    between paragraphs.
%
%    \begin{macrocode}
\usepackage{setspace}
%<thesis|minithesis|progress|project|report|gdp|gdpsummary>\onehalfspacing
%<article>\singlespacing
\setlength{\parindent}{0pt}
\setlength{\parskip}{2.0ex plus0.5ex minus0.2ex}
%    \end{macrocode}
%
%  \subsection{Margins}
%
%    Load the \texttt{vmargin} package to setup the page dimensions.
%
%    \begin{macrocode}
\usepackage{vmargin}
%<*thesis|minithesis|progress|project|report|gdp|gdpsummary>
\setmarginsrb           { 1.5in}  % left margin
                        { 0.6in}  % top margin
                        { 1.0in}  % right margin
                        { 0.8in}  % bottom margin
                        {  20pt}  % head height
                        {0.25in}  % head sep
                        {   9pt}  % foot height
                        { 0.3in}  % foot sep
%</thesis|minithesis|progress|project|report|gdp|gdpsummary>
%<*article>
\setmarginsrb           { 0.8in}  % left margin
                        { 0.6in}  % top margin
                        { 0.8in}  % right margin
                        { 0.8in}  % bottom margin
                        {  20pt}  % head height
                        {0.25in}  % head sep
                        {   9pt}  % foot height
                        { 0.3in}  % foot sep
%</article>
%    \end{macrocode}
%
% \subsection{Breaks}
%
%    Control how pages and lines are broken.
%
%    \begin{macrocode}
\raggedbottom
\setlength{\topskip}{1\topskip \@plus 5\p@}
\doublehyphendemerits=10000       % No consecutive line hyphens.
\brokenpenalty=10000              % No broken words across columns/pages.
\widowpenalty=9999                % Almost no widows at bottom of page.
\clubpenalty=9999                 % Almost no orphans at top of page.
\interfootnotelinepenalty=9999    % Almost never break footnotes.
%    \end{macrocode}
%
%  \subsection{Fancy Page Headers}
%
%    Setup the the default fancy page headers with an underlined header containing
%    chapter, name and page number.
%
%    \begin{macrocode}
\usepackage{fancyhdr}
\lhead[\rm\thepage]{\fancyplain{}{\sl{\rightmark}}}
\rhead[\fancyplain{}{\sl{\leftmark}}]{\rm\thepage}
\chead{}\lfoot{}\rfoot{}\cfoot{}
\pagestyle{fancy}
%    \end{macrocode}
%
% \begin{macro}{\chaptermark}
% \begin{macro}{\sectionmark}
% \begin{macro}{\subsectionmark}
%
%    Redefine \texttt{chaptermark} and \texttt{sectionmark} to set up the correct header name and produce a
%    banner on the output making the location of messages simpler.
%
%    \begin{macrocode}
%<*thesis|minithesis|progress|project|report|gdp>
\renewcommand{\chaptermark}[1]{\btypeout{\thechapter\space #1}\markboth{\@chapapp\ \thechapter\ #1}{\@chapapp\ \thechapter\ #1}}
\renewcommand{\sectionmark}[1]{}
\renewcommand{\subsectionmark}[1]{}
%</thesis|minithesis|progress|project|report|gdp>
%<*article>
\renewcommand{\sectionmark}[1]{\btypeout{\thesection\space #1}}
\renewcommand{\subsectionmark}[1]{}
\renewcommand{\subsubsectionmark}[1]{}
%</article>
%<gdpsummary>\markboth{GDP Summary Report}{GDP Summary Report}
%    \end{macrocode}
% \end{macro}
% \end{macro}
% \end{macro}
%
% \begin{macro}{\cleardoublepage}
%
%    Redefine \texttt{cleardoublepage} to remove headers from blank pages
%    in twosided documents.
%
%    \begin{macrocode}
\def\cleardoublepage{\clearpage\if@twoside \ifodd\c@page\else
\hbox{}
\thispagestyle{empty}
\newpage
\if@twocolumn\hbox{}\newpage\fi\fi\fi}
%    \end{macrocode}
% \end{macro}
%
% \subsection{Mathematics}
%
%    Load the \texttt{amsmath} packages, so we can do some serious math, and
%    setup the theorem environments.
%
%    \begin{macrocode}
\usepackage{amsmath,amsfonts,amssymb,amscd,amsthm,xspace}
\theoremstyle{plain}
%<*thesis|minithesis|progress|project|report|gdp>
\newtheorem{example}{Example}[chapter]
\newtheorem{theorem}{Theorem}[chapter]
%</thesis|minithesis|progress|project|report|gdp>
%<*article|gdpsummary>
\newtheorem{example}{Example}[section]
\newtheorem{theorem}{Theorem}[section]
%</article|gdpsummary>
\newtheorem{corollary}[theorem]{Corollary}
\newtheorem{lemma}[theorem]{Lemma}
\newtheorem{proposition}[theorem]{Proposition}
\newtheorem{axiom}[theorem]{Axiom}
\theoremstyle{definition}
\newtheorem{definition}[theorem]{Definition}
\theoremstyle{remark}
\newtheorem{remark}[theorem]{Remark}
%    \end{macrocode}
%
% \subsection{Captions}
%
% \begin{macro}{\fref}
% \begin{macro}{\tref}
% \begin{macro}{\eref}
% \begin{macro}{\cref}
% \begin{macro}{\sref}
% \begin{macro}{\aref}
%
%   Load the \texttt{caption2} package to improve formatting of captions,
%   and provide short referencing commands.
%
%    \begin{macrocode}
\usepackage[centerlast,small,sc]{caption2}
\setlength{\captionmargin}{20pt}
\newcommand{\fref}[1]{Figure~\ref{#1}}
\newcommand{\tref}[1]{Table~\ref{#1}}
\newcommand{\eref}[1]{Equation~\ref{#1}}
\newcommand{\cref}[1]{Chapter~\ref{#1}}
\newcommand{\sref}[1]{Section~\ref{#1}}
\newcommand{\aref}[1]{Appendix~\ref{#1}}
%    \end{macrocode}
%
% \end{macro}
% \end{macro}
% \end{macro}
% \end{macro}
% \end{macro}
% \end{macro}
%
% \subsection{Float Placement}
%
% \begin{macro}{\topfraction}
% \begin{macro}{\textfraction}
%    Preventing figures from appearing on a page by themselves. \LaTeX's figure
%    placement algorithm is quite biased in favour of putting figures on a page
%    by themselves, instead of on the top of a page with some text below it.
%    Fortunately, the parameters of the algorithm can be changed. The main problem
%    is that \LaTeX\ per default only allows a part of the top of a text-page (70\%)
%    to contain figures, and requires at least 20\% of a page to be text when text and
%    figures share a page. These parameters should be set to more reasonable values,
%    for example 85\% and 10\%.
%    \begin{macrocode}
\renewcommand{\topfraction}{0.85}
\renewcommand{\bottomfraction}{.85}
\renewcommand{\textfraction}{0.1}
\renewcommand{\dbltopfraction}{.85}
%    \end{macrocode}
% \end{macro}
% \end{macro}
% \begin{macro}{\floatpagefraction}
%    This helps, but sometimes \LaTeX\ puts a figure on a page by itself, although it
%    would fit perfectly well on the top of a page. This happens when the figure will
%    not fit on the page where it was defined. \LaTeX\ then attempts to put it on a
%    figures-only page before it attempts to put it at the top of the next page. A page
%    may contain figures alone if the figure(s) use at least half the page. To prevent
%    half-empty pages this limit should probably be increased to around 75\%.
%    \begin{macrocode}
\renewcommand{\floatpagefraction}{0.75}
\renewcommand{\dblfloatpagefraction}{.75}
%    \end{macrocode}
%    Be careful not to make \texttt{floatpagefraction} larger than \texttt{topfraction},
%    then you risk to produce a figure that can neither go on the top of a text page,
%    nor on a page by itself. If that happens, the figure and all later figures will be
%    postponed until next time a \texttt{clearpage} is executed (typically at the end of
%    a chapter or the end of the document). This will also happen if a figure is too
%    large to fit on a page.
%    \begin{macrocode}
\setcounter{topnumber}{9}
\setcounter{bottomnumber}{9}
\setcounter{totalnumber}{20}
\setcounter{dbltopnumber}{9}
%    \end{macrocode}
% \end{macro}
%
%
% \subsection{Graphics}
%
%    Load the \texttt{graphicx} package, so we can include pictures easily. Load the \texttt{epstopdf},
%    so that eps files are automatically converted to pdf when pdf\LaTeX\ is used. This requires a
%   version of {\em epstopdf} to be installed on you system. You can find a windows version at\newline
%   \texttt{http://www.ctan.org/tex-archive/support/epstopdf/epstopdf.exe}.
%
%    \begin{macrocode}
\usepackage{graphicx}
\usepackage{epstopdf}
%    \end{macrocode}
%
% \subsection{Subfigures and subtables}
%
%    Load the \texttt{subfigure} package, so we can include subfigures easily.
%
%    \begin{macrocode}
\usepackage[scriptsize]{subfigure}
\usepackage{booktabs}
\usepackage{rotating}
%    \end{macrocode}
%
% \subsection{Listings}
%
%    Load the \texttt{listings} package, so we can include listings easily.
%
%    \begin{macrocode}
\usepackage{listings}
\usepackage{lstpatch}
\lstset{captionpos=b,
        frame=tb,
        basicstyle=\scriptsize\ttfamily,
        showstringspaces=false,
        keepspaces=true}
\lstdefinestyle{matlab} {
        language=Matlab,
        keywordstyle=\color{blue},
        commentstyle=\color[rgb]{0.13,0.55,0.13}\em,
        stringstyle=\color[rgb]{0.7,0,0} }
%    \end{macrocode}
%
% \subsection{Hyperlinks}
%
%    Load the \texttt{hyperref} package to provide hyperlinks in pdf and dvi documents.
%
%    \begin{macrocode}
\usepackage[pdfpagemode={UseOutlines},bookmarks=true,bookmarksopen=true,
   bookmarksopenlevel=0,bookmarksnumbered=true,hypertexnames=false,
   colorlinks,linkcolor={blue},citecolor={blue},urlcolor={red},
   pdfstartview={FitV},unicode,breaklinks=true]{hyperref}
\pdfstringdefDisableCommands{
   \let\\\space
}
%    \end{macrocode}
%
% \section{Internal Names}
%
% \begin{macro}{\supervisor}
% \begin{macro}{\examiner}
% \begin{macro}{\degree}
% \begin{macro}{\authors}
% \begin{macro}{\addresses}
% \begin{macro}{\university}
% \begin{macro}{\UNIVERSITY}
% \begin{macro}{\department}
% \begin{macro}{\DEPARTMENT}
% \begin{macro}{\group}
% \begin{macro}{\GROUP}
% \begin{macro}{\faculty}
% \begin{macro}{\FACULTY}
% \begin{macro}{\subject}
% \begin{macro}{\keywords}
%
%    The various elements of the documents are defined
%    as control sequences to make it easy to customize this
%    style for other parts of the University.
%
%    \begin{macrocode}
\newcommand*{\supervisor}[1]{\def\supname{#1}}
\newcommand*{\examiner}[1]{\def\examname{#1}}
\newcommand*{\degree}[1]{\def\degreename{#1}}
\newcommand*{\authors}[1]{\def\authornames{#1}}
\newcommand*{\addresses}[1]{\def\addressnames{#1}}
\newcommand*{\university}[1]{\def\univname{#1}}
\newcommand*{\UNIVERSITY}[1]{\def\UNIVNAME{#1}}
\newcommand*{\department}[1]{\def\deptname{#1}}
\newcommand*{\DEPARTMENT}[1]{\def\DEPTNAME{#1}}
\newcommand*{\group}[1]{\def\groupname{#1}}
\newcommand*{\GROUP}[1]{\def\GROUPNAME{#1}}
\newcommand*{\faculty}[1]{\def\facname{#1}}
\newcommand*{\FACULTY}[1]{\def\FACNAME{#1}}
\newcommand*{\subject}[1]{\def\subjectname{#1}}
\newcommand*{\keywords}[1]{\def\keywordnames{#1}}
%    \end{macrocode}
%
% \end{macro}
% \end{macro}
% \end{macro}
% \end{macro}
% \end{macro}
% \end{macro}
% \end{macro}
% \end{macro}
% \end{macro}
% \end{macro}
% \end{macro}
% \end{macro}
% \end{macro}
% \end{macro}
% \end{macro}
%
%
% \begin{macro}{\supname}
% \begin{macro}{\examname}
% \begin{macro}{\degreename}
% \begin{macro}{\authornames}
% \begin{macro}{\univname}
% \begin{macro}{\UNIVNAME}
% \begin{macro}{\deptname}
% \begin{macro}{\DEPTNAME}
% \begin{macro}{\groupname}
% \begin{macro}{\GROUPNAME}
% \begin{macro}{\facname}
% \begin{macro}{\FACNAME}
% \begin{macro}{\subjectname}
% \begin{macro}{\keywordnames}
%
%    The internal names of the elements are set
%    to defaults appropriate to the Department of Electronics and Computer Science.
%
%    \begin{macrocode}
\supervisor  {}
\examiner    {}
\degree      {}
\authors     {}
\university  {\texorpdfstring{\href{http://www.soton.ac.uk}
                {University of Southampton}}
                {University of Southampton}}
\UNIVERSITY  {\texorpdfstring{\href{http://www.soton.ac.uk}
                {UNIVERSITY OF SOUTHAMPTON}}
                {UNIVERSITY OF SOUTHAMPTON}}
\department  {\texorpdfstring{\href{http://www.ecs.soton.ac.uk}
                {School of Electronics and Computer Science}}
                {School of Electronics and Computer Science}}
\DEPARTMENT  {\texorpdfstring{\href{http://www.ecs.soton.ac.uk}
                {SCHOOL OF ELECTRONICS AND COMPUTER SCIENCE}}
                {SCHOOL OF ELECTRONICS AND COMPUTER SCIENCE}}
\group       {\texorpdfstring{\href{http://www.isis.ecs.soton.ac.uk}
                {Image, Speech and Intelligent Systems Group}}
                {Image, Speech and Intelligent Systems Group}}
\GROUP       {\texorpdfstring{\href{http://www.isis.ecs.soton.ac.uk}
                {IMAGE, SPEECH AND INTELLIGENT SYSTEMS GROUP}}
                {IMAGE, SPEECH AND INTELLIGENT SYSTEMS GROUP}}
\faculty     {\texorpdfstring{\href{http://www.engineering.soton.ac.uk}
                {Faculty of Engineering and Physical Sciences}}
                {Faculty of Engineering and Physical Sciences}}
\FACULTY     {\texorpdfstring{\href{http://www.engineering.soton.ac.uk}
                {FACULTY OF ENGINEERING, SCIENCE AND MATHEMATICS}}
                {FACULTY OF ENGINEERING, SCIENCE AND MATHEMATICS}}
\addresses   {}
\subject     {}
\keywords    {}
%    \end{macrocode}
%
% \end{macro}
% \end{macro}
% \end{macro}
% \end{macro}
% \end{macro}
% \end{macro}
% \end{macro}
% \end{macro}
% \end{macro}
% \end{macro}
% \end{macro}
% \end{macro}
% \end{macro}
% \end{macro}
%
% \section{Special Pages}
%
% \subsection{Title Page}
%
% \begin{macro}{\maketitle}
%
%    Setup an appropriate title page and pdf strings if we are using
%    pdf\LaTeX.
%
%   \begin{macrocode}
%<*thesis|minithesis|progress|project|report|gdp|gdpsummary>}
\renewcommand\maketitle{
  \btypeout{Title Page}
  \hypersetup{pdftitle={\@title}}
  \hypersetup{pdfsubject=\subjectname}
  \hypersetup{pdfauthor=\authornames}
  \hypersetup{pdfkeywords=\keywordnames}
  \thispagestyle{empty}
  \begin{titlepage}
    \let\footnotesize\small
    \let\footnoterule\relax
    \let \footnote \thanks
    \setcounter{footnote}{0}
    \null\vfil
    \vskip 60\p@
    \begin{center}
      \setlength{\parskip}{0pt}
      {\large\textbf{\UNIVNAME}\par}
%<*minithesis|progress|project|gdp|gdpsummary>
      {\large \facname \par}
%</minithesis|progress|project|gdp|gdpsummary>
%<*minithesis|progress|project>
      {\large \deptname \par}
%</minithesis|progress|project>
      \vfill
%<*minithesis>
      {\large A mini-thesis submitted for transfer from}
      {\large MPhil to PhD \par}
%</minithesis>
%<*progress>
      {\large A progress report submitted for continuation}
      {\large towards a PhD \par}
%</progress>
%<*project>
      {\large A project report submitted for the}
      {\large award of \par \degreename \par}
%</project>
%<*gdp>
      {\large A group design project report submitted for}
      {\large the award of \par \degreename \par}
%</gdp>
%<*gdpsummary>
      {\large A group design project summary report submitted}
      {\large for the award of \par \degreename \par}
%</gdpsummary>
%<*minithesis|progress|project|gdp|gdpsummary>
      \vfill
      {\normalsize Supervisor: \supname \par}
      {\normalsize Examiner: \examname \par}
      \vfill
      \hspace{6mm}\parbox[t][51mm][s]{89mm}{
        \center
        \vfill
        {\large \bf \@title \par}
        \vfill
        {\normalsize by \authornames \par}
        \vfill
        {\normalsize \@date \par}
        \vfill
      }
      \parbox[t][95mm][s]{89mm}{}
%</minithesis|progress|project|gdp|gdpsummary>
%<*thesis|report>
      {\huge \bf \@title \par}
      \vfill
      {\LARGE by \par}
      \smallskip
      {\LARGE \authornames \par}
      \vfill
%<*thesis>
      {\large A thesis submitted in partial fulfillment for the \par}
      {\large degree of Doctor of Philosophy \par}
%</thesis>
%<report>      {\large Technical Report \par}
      \bigskip
      \bigskip
%<thesis>      {\large in the \par}
      {\large \facname \par}
      {\large \deptname \par}
      \bigskip
      \bigskip
      \bigskip
      {\Large \@date \par}
      \bigskip
%</thesis|report>
    \end{center}
    \par
    \@thanks
    \vfil\null
  \end{titlepage}
  \setcounter{footnote}{0}%
  \global\let\thanks\relax
  \global\let\maketitle\relax
  \global\let\@thanks\@empty
  \global\let\@author\@empty
  \global\let\@date\@empty
  \global\let\@title\@empty
  \global\let\title\relax
  \global\let\author\relax
  \global\let\date\relax
  \global\let\and\relax
  \cleardoublepage
}
%</thesis|minithesis|progress|project|report|gdp|gdpsummary>
%<*article>
\if@titlepage
  \renewcommand\maketitle{
    \btypeout{Title Page}
    \hypersetup{pdftitle={\@title}}
    \hypersetup{pdfsubject={\subjectname}}
    \hypersetup{pdfauthor={\authornames}}
    \hypersetup{pdfkeywords={\keywordnames}}
    \markboth{\authornames}{\@title}
    \begin{titlepage}
    \thispagestyle{empty}
    \let\footnotesize\small
    \let\footnoterule\relax
    \let \footnote \thanks
    \null\vfil
    \vskip 60\p@
    \begin{center}
      {\LARGE \@title \par}
      \vskip 3em
      {\large \lineskip .75em
        \begin{tabular}[t]{c} \authornames \end{tabular}
      \par}
      \vskip 1em
      {\large \lineskip .5em
        \begin{tabular}[t]{c} \addressnames \end{tabular}
      \par}
      \vskip 1.5em
      {\large \@date \par}
      \if\keywordnames
      \else
        \quotation
        \vskip 1.5em
        {\noindent \normalsize
            \textbf{Keywords:}
            \textit{\keywordnames}
        \par}
        \endquotation
      \fi
    \end{center}
    \par
    \@thanks
    \vfil\null
    \end{titlepage}
    \setcounter{footnote}{0}
    \global\let\thanks\relax
    \global\let\maketitle\relax
    \global\let\@maketitle\relax
    \global\let\@thanks\@empty
    \global\let\@author\@empty
    \global\let\@date\@empty
    \global\let\@title\@empty
    \global\let\title\relax
    \global\let\author\relax
    \global\let\date\relax
    \global\let\and\relax
  }
\else
  \renewcommand\maketitle{
    \btypeout{Title Page}
    \hypersetup{pdftitle={\@title}}
    \hypersetup{pdfsubject=\subjectname}
    \hypersetup{pdfauthor=\authornames}
    \hypersetup{pdfkeywords=\keywordnames}
    \markboth{\authornames}{\@title}
    \thispagestyle{empty}
    \par
    \begingroup
      \renewcommand\thefootnote{\@fnsymbol\c@footnote}
      \def\@makefnmark{
        \rlap{\@textsuperscript{\normalfont\@thefnmark}}
      }
      \long\def\@makefntext##1{
        \parindent 1em\noindent \hb@xt@1.8em
        {\hss\@textsuperscript{\normalfont\@thefnmark}}##1
      }
      \if@twocolumn
        \ifnum \col@number=\@ne
          \@maketitle
        \else
          \twocolumn[\@maketitle]%
        \fi
      \else
        \newpage
        % Prevent figures from going at top of page.
        \global\@topnum\z@
        \@maketitle
      \fi
      \thispagestyle{plain}\@thanks
    \endgroup
    \setcounter{footnote}{0}%
    \global\let\thanks\relax
    \global\let\maketitle\relax
    \global\let\@maketitle\relax
    \global\let\@thanks\@empty
    \global\let\@author\@empty
    \global\let\@date\@empty
    \global\let\@title\@empty
    \global\let\title\relax
    \global\let\author\relax
    \global\let\date\relax
    \global\let\and\relax
  }
  \def\@maketitle{
    \newpage
    \null
    \vskip 2em
    \begin{center}
      \let \footnote \thanks
      {\LARGE \@title \par}
      \vskip 1.5em
      {\large \lineskip .5em
        \begin{tabular}[t]{c} \authornames \end{tabular}
      \par}
      \vskip 0.7em
      {\large \lineskip .5em
        \begin{tabular}[t]{c} \addressnames \end{tabular}
      \par}
      \vskip 1em
      {\large \@date}
    \end{center}
    \par
    \vskip 1.5em
  }
\fi
%</article>
%    \end{macrocode}
%
% \end{macro}
%
% \subsection{Abstract Page}
%
% \begin{macro}{\abstract}
%
%    Setup an appropriate abstract page.
%
%    \begin{macrocode}
%<*thesis|minithesis|progress|project|report|gdp>
\newenvironment{abstract}
{
  \btypeout{Abstract Page}
  \thispagestyle{empty}
  \null\vfil
  \begin{center}
    \setlength{\parskip}{0pt}
    {\normalsize \UNIVNAME \par}
    \bigskip
    {\underline{ABSTRACT} \par}
    \bigskip
    {\normalsize \FACNAME \par}
%<*thesis|minithesis|progress|project|report>
    {\normalsize \DEPTNAME \par}
%</thesis|minithesis|progress|project|report>
    \bigskip
%<*thesis>
    {\normalsize \underline{Doctor of Philosophy}\par}
%</thesis>
%<*minithesis>
    {\normalsize \underline{A mini-thesis submitted for transfer from MPhil to PhD}\par}
%</minithesis>
%<*progress>
    {\normalsize \underline{A progress report submitted for continuation towards a PhD}\par}
%</progress>
%<*project>
    {\normalsize \underline{A project report submitted for the award of \degreename}\par}
%</project>
%<*gdp>
    {\normalsize \underline{A group design project report submitted for the award of \degreename}\par}
%</gdp>
    \bigskip
    {\normalsize\bf \@title \par}
    \medskip
    {\normalsize by \authornames \par}
    \bigskip
  \end{center}
}
{
  \vfil\vfil\vfil\null
  \cleardoublepage
}
%</thesis|minithesis|progress|project|report|gdp>
%<*article>
\if@titlepage
  \renewenvironment{abstract}{
      \titlepage
      \null\vfil
      \@beginparpenalty\@lowpenalty
      \begin{center}
        \bfseries \abstractname
        \@endparpenalty\@M
      \end{center}
      \begin{itshape}
      \noindent
  }
  {
      \par
      \end{itshape}
      \if\keywordnames
      \else
        \quote
        \vskip 1.5em
        {\noindent \normalsize
            \textbf{Keywords:}
            \textit{\keywordnames}
        \par}
        \endquote
      \fi
      \vfil\null\endtitlepage
  }
\else
  \renewenvironment{abstract}{
      \if@twocolumn
        \section*{\abstractname}
      \else
        \small
        \begin{center}
          {\bfseries \abstractname\vspace{-.5em}\vspace{\z@}}
        \end{center}
        \quote
      \fi
      \begin{itshape}
  }
  {
      \end{itshape}
      \if\keywordnames
      \else
        \vskip 1.5em
        {\noindent \normalsize
            \textbf{Keywords:}
            \textit{\keywordnames}
        \par}
      \fi
      \if@twocolumn\else\endquote\fi
  }
\fi
%</article>
%    \end{macrocode}
%
% \end{macro}
%
% \subsection{Contents Pages}
%
% \subsubsection{Add to Table of Contents}
%
%    Ensures numbering for sub-subsections in the table of contents, and provide
%    for 6 levels to appear in toc. Define \texttt{addtotoc} to enable adding
%    elements to the toc at chapter level, using a dummy counter to fix bookmarks
%    in pdf\LaTeX.
%
%    \begin{macrocode}
%<*thesis|minithesis|progress|project|report|gdp|article>
\addtocounter{secnumdepth}{1}
\setcounter{tocdepth}{6}
\newcounter{dummy}
\newcommand\addtotoc[1]{
\refstepcounter{dummy}
%<*thesis|minithesis|progress|project|report|gdp>
\addcontentsline{toc}{chapter}{#1}}
%</thesis|minithesis|progress|project|report|gdp>
%<*article>
\addcontentsline{toc}{section}{#1}}
%</article>
%    \end{macrocode}
%
% \subsubsection{Table of Contents}
%
% \begin{macro}{\tableofcontents}
%
%    Adjust line and paragraph spacing.
%
%    \begin{macrocode}
\renewcommand\tableofcontents{
\btypeout{Table of Contents}
\begin{spacing}{1}{
    \setlength{\parskip}{1pt}
    \if@twocolumn
      \@restonecoltrue\onecolumn
    \else
      \@restonecolfalse
    \fi
%<*thesis|minithesis|progress|project|report|gdp>
    \chapter*{\contentsname
%</thesis|minithesis|progress|project|report|gdp>
%<*article>
    \section*{\contentsname
%</article>
        \@mkboth{
           \MakeUppercase\contentsname}{\MakeUppercase\contentsname}}
    \@starttoc{toc}
    \if@restonecol\twocolumn\fi
%<thesis|minithesis|progress|project|report|gdp>   \cleardoublepage
}\end{spacing}
}
%    \end{macrocode}
% \end{macro}
%
% \subsubsection{List of Figures}
%
% \begin{macro}{\listoffigures}
%
%    Adjust line and paragraph spacing.
%
%    \begin{macrocode}
\renewcommand\listoffigures{
\btypeout{List of Figures}
\begin{spacing}{1}{
    \setlength{\parskip}{1pt}
    \if@twocolumn
      \@restonecoltrue\onecolumn
    \else
      \@restonecolfalse
    \fi
%<*thesis|minithesis|progress|project|report|gdp>
    \chapter*{\listfigurename
%</thesis|minithesis|progress|project|report|gdp>
%<*article>
    \section*{\listfigurename
%</article>
      \@mkboth{\MakeUppercase\listfigurename}
              {\MakeUppercase\listfigurename}}
    \@starttoc{lof}
    \if@restonecol\twocolumn\fi
%<*thesis|minithesis|progress|project|report|gdp>
    \cleardoublepage
%</thesis|minithesis|progress|project|report|gdp>
}\end{spacing}
}
%    \end{macrocode}
% \end{macro}
%
% \subsubsection{List of Tables}
%
% \begin{macro}{\listoftables}
%
%    Adjust line and paragraph spacing.
%
%    \begin{macrocode}
\renewcommand\listoftables{
\btypeout{List of Tables}
\begin{spacing}{1}{
    \setlength{\parskip}{1pt}
    \if@twocolumn
      \@restonecoltrue\onecolumn
    \else
      \@restonecolfalse
    \fi
%<*thesis|minithesis|progress|project|report|gdp>
    \chapter*{\listtablename
%</thesis|minithesis|progress|project|report|gdp>
%<*article>
    \section*{\listtablename
%</article>
      \@mkboth{
          \MakeUppercase\listtablename}{\MakeUppercase\listtablename}}
    \@starttoc{lot}
    \if@restonecol\twocolumn\fi
%<*thesis|minithesis|progress|project|report|gdp>
    \cleardoublepage
%</thesis|minithesis|progress|project|report|gdp>
}\end{spacing}
}
%    \end{macrocode}
% \end{macro}
%
% \subsection{List of Symbols}
%
% \begin{macro}{\listofsymbols}
%
%    Provide a \texttt{listofsymbols} command to produce a list of symbols,
%
%    \begin{macrocode}
%<*thesis|minithesis|progress|report|article>
\newcommand\listsymbolname{Nomenclature}
%</thesis|minithesis|progress|report|article>
%<*project|gdp>
\newcommand\listsymbolname{List of Symbols}
%</project|gdp>
\usepackage{longtable}
\newcommand\listofsymbols[2]{
\btypeout{\listsymbolname}
\addtotoc{\listsymbolname}
%<*thesis|minithesis|progress|project|report|gdp>
    \chapter*{\listsymbolname
%</thesis|minithesis|progress|project|report|gdp>
%<*article>
    \section*{\listsymbolname
%</article>
      \@mkboth{
          \MakeUppercase\listsymbolname}{\MakeUppercase\listsymbolname}}
\begin{longtable}[c]{#1}#2\end{longtable}\par
%<*thesis|minithesis|progress|project|report|gdp>
    \cleardoublepage
%</thesis|minithesis|progress|project|report|gdp>
}
%</thesis|minithesis|progress|project|report|gdp|article>
%    \end{macrocode}
%
% \end{macro}
%
% \subsection{Acknowledgements}
%
% \begin{macro}{\acknowledgements}
%
%    Thank someone!
%
%    \begin{macrocode}
\newcommand\acknowledgements[1]{
\btypeout{Acknowledgements}
\addtotoc{Acknowledgements}
%<*thesis|minithesis|progress|project|report|gdp>
\thispagestyle{plain}
\null\vfil
\vskip 60\p@
\begin{center}{\Large\bf Acknowledgements \par}\end{center}
{\normalsize #1}
\vfil\vfil\null
\cleardoublepage
%</thesis|minithesis|progress|project|report|gdp>
%<*article>
\section*{Acknowledgements}
{\normalsize #1}
%</article>
}
%    \end{macrocode}
%
% \end{macro}
%
% \subsection{Dedicatory}
%
% \begin{macro}{\dedicatory}
%
%    Dedicate the document to someone!
%
%    \begin{macrocode}
\newcommand\dedicatory[1]{
\btypeout{Dedicatory}
%<*thesis|minithesis|progress|project|report|gdp>
\thispagestyle{plain}
\null\vfil
\vskip 60\p@
\begin{center}{\Large \sl #1}\end{center}
\vfil\null
\cleardoublepage
%</thesis|minithesis|progress|project|report|gdp>
}
%    \end{macrocode}
%
% \end{macro}
%
% \section{Parts}
%
% \begin{macro}{\frontmatter}
%
%    Modify \texttt{frontmatter}.
%
%    \begin{macrocode}
%<*article>
\newcommand\frontmatter{ }
%</article>
%    \end{macrocode}
% \end{macro}
%
% \begin{macro}{\mainmatter}
%
%    Modify \texttt{mainmatter}.
%
%    \begin{macrocode}
%<*article>
\newcommand\mainmatter{ }
%</article>
%    \end{macrocode}
% \end{macro}
%
% \begin{macro}{\backmatter}
%
%    Modify \texttt{backmatter} to add the bibliography to the table of contents.
%
%    \begin{macrocode}
%<*thesis|minithesis|progress|project|report|gdp>
\renewcommand\backmatter{
  \if@openright
    \cleardoublepage
  \else
    \clearpage
  \fi
  \addtotoc{\bibname}
  \btypeout{\bibname}
  \@mainmatterfalse}
%</thesis|minithesis|progress|project|report|gdp>
%<*article>
\newcommand\bibname{References}
\newcommand\backmatter{
  \addtotoc{\bibname}
  \btypeout{\bibname}
}
%</article>
%    \end{macrocode}
% \end{macro}
%
%    \begin{macrocode}
%</thesis|minithesis|progress|project|report|article|gdp|gdpsummary>
%    \end{macrocode}
%
%
% \section{Example Documents}
%
%    \begin{macrocode}
%<*testthesis>
\documentclass{ecsthesis}      % Use the Thesis Style
%</testthesis>
%<*testminithesis>
\documentclass{ecsminithesis}  % Use the minithesis Style
%</testminithesis>
%<*testprogress>
\documentclass{ecsprogress}    % Use the progress Style
%</testprogress>
%<*testproject>
\documentclass{ecsproject}     % Use the Project Style
%</testproject>
%<*testreport>
\documentclass{ecsreport}      % Use the Report Style
%</testreport>
%<*testarticle>
\documentclass{ecsarticle}     % Use the Article Style
%</testarticle>
%<*testgdp>
\documentclass{ecsgdp}         % Use the GDP Report Style
%</testgdp>
%<*testgdpsummary>
\documentclass{ecsgdpsummary}  % Use the GDP Summary Report Style
%</testgdpsummary>
%<*testthesis|testminithesis|testprogress|testproject|testreport|testarticle|testgdp|testgdpsummary>
\graphicspath{{../Figures/}}   % Location of your graphics files
\usepackage{natbib}            % Use Natbib style for the refs.
\hypersetup{colorlinks=true}   % Set to false for black/white printing
%% ----------------------------------------------------------------
%% Definitions.tex
%% ---------------------------------------------------------------- 
\newcommand{\BibTeX}{{\rm B\kern-.05em{\sc i\kern-.025em b}\kern-.08em T\kern-.1667em\lower.7ex\hbox{E}\kern-.125emX}}

%% People
\newcounter{address}
\setcounter{address}{1}
\renewcommand{\theaddress}{\textsuperscript{\fnsymbol{address}}}
\newcommand{\address}[1]{\refstepcounter{address}\theaddress#1\\}
\newcommand{\Name}[3]{\texorpdfstring{\href{mailto:#3}{#2}#1}{#2}\xspace}
\newcommand{\Adrian}[1]{\Name{#1}{Adrian Azzarelli}{aab1g18@soton.ac.uk}}

%% Dingbats
\newcommand{\tick}{\ding{51}}
\newcommand{\cross}{\ding{55}}

%% Calculus
\newcommand{\pd}[2]{\ensuremath{\frac{\partial #1}{\partial #2}}\xspace}
\newcommand{\fd}[2]{\ensuremath{\frac{d #1}{d #2}}\xspace}
\newcommand{\dint}{\ensuremath{\int\!\!\!\int}\xspace}
\newcommand{\tint}{\ensuremath{\int\!\!\!\int\!\!\!\int}\xspace}

%% Math Sets
\newcommand{\Q}[1]{\ensuremath{\mathbb{#1}}\xspace}
\newcommand{\R}{\Q{R}}

%% Matrix, Vector
\newcommand{\V}[1]{\ensuremath{\boldsymbol{#1}}\xspace}
\newcommand{\M}[1]{\ensuremath{\boldsymbol{#1}}\xspace}
\newcommand{\0}{\V{0}}
\newcommand{\1}{\V{1}}
\newcommand{\I}{\M{I}}

%% Math Functions
\newcommand{\F}[1]{\ensuremath{\mathrm{#1}}\xspace}
\newcommand{\sgn}{\F{sgn}}
\newcommand{\tr}{\F{trace}}
\newcommand{\diag}{\F{diag}}

%% Math Names
\newcommand{\N}[1]{\ensuremath{\mathit{#1}}\xspace}

%% Data
\newcommand{\mc}[1]{\ensuremath{\mathcal{#1}}\xspace}
\newcommand{\Hyp}{\mc{H}}
\newcommand{\D}{\mc{D}}

%% Kernel
\newcommand{\K}{\M{K}}
\newcommand{\eins}{\texorpdfstring{\ensuremath{\epsilon}}{\textepsilon}-insensitive\xspace}
\newcommand{\e}{\ensuremath{\epsilon}\xspace}
\newcommand{\Bxi}{\ensuremath{\boldsymbol{\xi}}\xspace}
\newcommand{\Kanova}{\ensuremath{\mathit{K_{ANOVA}}}\xspace}
\newcommand{\Kspline}{\ensuremath{\mathit{K_{spline}}}\xspace}

%% Bayesian
\newcommand{\MP}{\ensuremath{\mathit{{\scriptscriptstyle \hspace{-1.5pt}M\hspace{-1.5pt}P}}}\xspace}
\newcommand{\ML}{\ensuremath{\mathit{{\scriptscriptstyle \hspace{-1.5pt}M\hspace{-1.5pt}L}}}\xspace}
\newcommand{\Qw}{\ensuremath{Q_{\w}(\w)}\xspace}
\newcommand{\Qa}{\ensuremath{Q_{\Ba}(\Ba)}\xspace}
\newcommand{\Qb}{\ensuremath{Q_{\beta}(\beta)}\xspace}
\newcommand{\wMPab}{\ensuremath{\w_{\MP|\bar {\Ba},\bar \beta}}\xspace}
\newcommand{\wMP}{\ensuremath{\w_{\MP}}\xspace}
\newcommand{\yMP}{\ensuremath{y_{\MP}}\xspace}
\newcommand{\BaMP}{\ensuremath{\Ba_{\hspace{1pt}\MP}}\xspace}
\newcommand{\aMP}{\ensuremath{\alpha_{\hspace{1pt}\MP}}\xspace}
\newcommand{\bMP}{\ensuremath{\beta_{\hspace{1pt}\MP}}\xspace}
\newcommand{\Sab}{\ensuremath{\M{\Sigma}_{\bar \Ba,\bar \beta}}\xspace}
\newcommand{\Ba}{\ensuremath{\boldsymbol{\alpha}}\xspace}
\newcommand{\Bb}{\ensuremath{\boldsymbol{\beta}}\xspace}
\newcommand{\Bm}{\ensuremath{\boldsymbol{\mu}}\xspace}
\newcommand{\BL}{\ensuremath{\boldsymbol{\Lambda}}\xspace}
\newcommand{\BPhi}{\ensuremath{\boldsymbol{\Phi}}\xspace}
\newcommand{\SMP}{\ensuremath{\M{\Sigma}_{\MP}}\xspace}

\newcommand{\Pa}{\ensuremath{P(\alpha|\mathcal{H})}\xspace}
\newcommand{\Pb}{\ensuremath{P(\beta|\mathcal{H})}\xspace}
\newcommand{\Pab}{\ensuremath{P(\alpha,\beta|\mathcal{H})}\xspace}
\newcommand{\Pw}{\ensuremath{P(\w|\mathcal{H})}\xspace}
\newcommand{\PD}{\ensuremath{P(\D|\mathcal{H})}\xspace}
\newcommand{\PwIa}{\ensuremath{P(\w|\alpha,\mathcal{H})}\xspace}
\newcommand{\PDIwb}{\ensuremath{P(\D|\w,\beta,\mathcal{H})}\xspace}
\newcommand{\PDwab}{\ensuremath{P(\D,\w,\alpha,\beta|\mathcal{H})}\xspace}
\newcommand{\PDIw}{\ensuremath{P(\D|\w,\mathcal{H})}\xspace}
\newcommand{\PwID}{\ensuremath{P(\w|\D,\mathcal{H})}\xspace}
\newcommand{\PwabID}{\ensuremath{P(\w,\alpha,\beta|\D,\mathcal{H})}\xspace}

\newcommand{\PanH}{\ensuremath{P(\alpha)}\xspace}
\newcommand{\PbnH}{\ensuremath{P(\beta)}\xspace}
\newcommand{\PabnH}{\ensuremath{P(\alpha,\beta)}\xspace}
\newcommand{\PwnH}{\ensuremath{P(\w)}\xspace}
\newcommand{\PDnH}{\ensuremath{P(\D)}\xspace}
\newcommand{\PwIanH}{\ensuremath{P(\w|\alpha)}\xspace}
\newcommand{\PDIwbnH}{\ensuremath{P(\D|\w,\beta)}\xspace}
\newcommand{\PDwabnH}{\ensuremath{P(\D,\w,\Ba,\beta)}\xspace}
\newcommand{\PDIwnH}{\ensuremath{P(\D|\w)}\xspace}
\newcommand{\PwIDnH}{\ensuremath{P(\w|\D)}\xspace}
\newcommand{\PwabIDnH}{\ensuremath{P(\w,\alpha,\beta|\D)}\xspace}

\newcommand{\PDwBab}{\ensuremath{P(\D,\w,\Ba,\beta|\mathcal{H})}\xspace}
\newcommand{\PwIBa}{\ensuremath{P(\w|\Ba,\mathcal{H})}\xspace}
\newcommand{\PBab}{\ensuremath{P(\Ba,\beta|\mathcal{H})}\xspace}
\newcommand{\PwBabID}{\ensuremath{P(\w,\Ba,\beta|\D,\mathcal{H})}\xspace}

\newcommand{\PBanH}{\ensuremath{P(\Ba)}\xspace}
\newcommand{\PwIBanH}{\ensuremath{P(\w|\Ba)}\xspace}

%% Snakes
\newcommand{\Esnake}{\ensuremath{\mathit{E_{snake}}}\xspace}
\newcommand{\Eimage}{\ensuremath{\mathit{E_{image}}}\xspace}
\newcommand{\Econt}{\ensuremath{\mathit{E_{cont}}}\xspace}
\newcommand{\Ecurv}{\ensuremath{\mathit{E_{curv}}}\xspace}
\newcommand{\Eint}{\ensuremath{\mathit{E_{int}}}\xspace}
\newcommand{\Eext}{\ensuremath{\mathit{E_{ext}}}\xspace}
\newcommand{\Eterm}{\ensuremath{\mathit{E_{term}}}\xspace}
\newcommand{\Eline}{\ensuremath{\mathit{E_{line}}}\xspace}
\newcommand{\Eedge}{\ensuremath{\mathit{E_{edge}}}\xspace}
\newcommand{\Econ}{\ensuremath{\mathit{E_{con}}}\xspace}
\newcommand{\Eangle}{\ensuremath{\mathit{E_{angle}}}\xspace}
\newcommand{\Elshape}{\ensuremath{\mathit{E_{lshape}}}\xspace}
\newcommand{\Eedgedir}{\ensuremath{\mathit{E_{edgedir}}}\xspace}
\newcommand{\Emodel}{\ensuremath{\mathit{E_{model}}}\xspace}
\newcommand{\wte}{\ensuremath{\mathit{w_{term}}}\xspace}
\newcommand{\wli}{\ensuremath{\mathit{w_{line}}}\xspace}
\newcommand{\wed}{\ensuremath{\mathit{w_{edge}}}\xspace}
\newcommand{\wco}{\ensuremath{\mathit{w_{con}}}\xspace}

%% Environments
\newcounter{alg}
\newenvironment{algorithm}[1]
{
    \stepcounter{alg}
    \begin{table}[htb]
    \centering
    \begin{tabular}[t]{ll}
    \hline&\\
    \multicolumn{2}{l}{\bf Algorithm \arabic{alg}: #1}\\&\\
} {
    &\\
    \hline
    \end{tabular}
    \end{table}
}
            % Include your abbreviations
%% ----------------------------------------------------------------
\begin{document}
\frontmatter
\title      {An Investigation into \dots}
\authors    {\texorpdfstring
             {\href{mailto:S.R.Gunn@ecs.soton.ac.uk}{Steve R. Gunn}}
             {Steve R. Gunn}
            }
\addresses  {\groupname\\\deptname\\\univname}
\date       {\today}
\subject    {}
\keywords   {}
%<*testminithesis|testprogress|testproject|testgdp|testgdpsummary>
\supervisor {Mr Beavis}
\examiner   {Dr Butthead}
%</testminithesis|testprogress|testproject|testgdp|testgdpsummary>
%<*testproject>
\degree     {BEng Electronic Engineering}
%</testproject>
%<*testgdp|testgdpsummary>
\degree     {Master of Engineering}
%</testgdp|testgdpsummary>
\maketitle
%</testthesis|testminithesis|testprogress|testproject|testreport|testarticle|testgdp|testgdpsummary>
%<*testthesis|testminithesis|testprogress|testproject|testreport|testarticle|testgdp>
\begin{abstract}
This work is all about \dots
\end{abstract}
\tableofcontents
\listoffigures
\listoftables
\lstlistoflistings
\listofsymbols{ll}{$w$ & The weight vector}
%</testthesis|testminithesis|testprogress|testproject|testreport|testarticle|testgdp>
%<*testthesis|testminithesis|testprogress|testproject|testreport|testgdp>
\acknowledgements{Thanks to no one.}
\dedicatory{To \dots}
%</testthesis|testminithesis|testprogress|testproject|testreport|testgdp>
%<*testthesis|testminithesis|testprogress|testproject|testreport|testarticle|testgdp|testgdpsummary>
\mainmatter
%% ----------------------------------------------------------------
%</testthesis|testminithesis|testprogress|testproject|testreport|testarticle|testgdp|testgdpsummary>
%<*testthesis|testminithesis|testprogress|testproject|testreport|testgdp>
%% ----------------------------------------------------------------
%% Introduction.tex
%% ---------------------------------------------------------------- 
\chapter{Introduction} \label{ch:intro}
\section{Introduction}\label{ch:intro.intro}
The goal of this project is to propose a philosophical framework that incorporates artificial intelligence methods with the aim of assisting a (human) \textit{client} in making a decision given an ethical dilemma. To accomplish this, several intermediary decisions regarding the potential outcomes will be made relative to selected philosophical perspectives, for example the \textit{moral perspective} will supply a morally-just decision, whereas the \textit{social perspective} will provide a socially-just decision. Thereafter amalgamating decisions, a final (single) decision will be determined reasonably, thus the assistant provides the client with a suitable choice of outcome. 

The two prevalent topics of research explored within this report consist of, the subject of philosophy investigated in Chapter~\ref{ch:philo}, which outlines the philosophical properties that my decision-making (DM) tool will incorporate, and technological modelling explored in Chapter~\ref{ch:tech}, which underpins the DM techniques I have applied to my decision-making model.

Much of the (philosophical) literature around decision making focuses on several branches of philosophy: (1)\textit{Meta-ethics}, the study of morality, as \cite{miner2003moral} outlines there is a "need for training in moral philosophy"\footnote{In the context of the impact of \textit{Moral Theory} on decision making models}, (2)\textit{Social influence}, described by \cite{baqer2012tech} as the ethics that "change with the trends of society", and (3)\textit{Legal influence}, which is acknowledged by \cite{dennis2013ethical} as important in any autonomous system, though can be disregarded when illegalities can not be avoided or when other ethics take precedent. I have labelled these (Metaethics, Sociology and Legality) as the \textbf{pillars} to my DM model, where (2) holds the most weight, given a study from \cite{nolan2008normative} supports\textit{ descriptive normative beliefs}\footnote{\textit{Descriptive normative beliefs} refer to what an individual think the social norm is} as the most influential behaviour\footnote{Interestingly the study exhibits peoples absent-thought towards the influence of normative action} and (1) would hold the second rank, as I have reasoned with \cite{dennis2013ethical} behind the lowered importance of legal impact.

In addition to layering my model with philosophical reasoning, I need to apply a technical framework that supports it. There are several popular structures of DM models, such as \textit{priority queues}\footnote{A \textit{priority queue} is a selection sorting algorithm that assigns priorities to each outcome and decides on the highest-priority item}, used in \cite{islam2018algorithm} as a \textit{heap} (tree-like structure), or \textit{cost structures}\footnote{A \textit{cost structure} is a model that incorporates a cost-function with the purpose of finding the least-cost outcome}, explored in \cite{lee1999mathematical} as a \textit{centralised system}\footnote{A \textit{centralised system} incorporates a central component that relays with external components individually. The alternative is a decentralised system,  where all components relay with each other (similar to a fully-connected graph), though the central component still makes the primary decisions}. There also exists purely learning structures (relating to artificial intelligence (AI)), such as in \cite{noothigattu2017voting}, who uses a voting-based system to train decisions, and \cite{KARTAL2016599} who finds algorithms such as support vector machine (SVM) algorithms useful for multi-criteria decision making tasks in cars.

Aside from structuring my framework, I need to hurdle two obstacles in order to develop an exemplar program. The first focuses on the translation of philosophical topics and contextual information\footnote{\textit{Contextual information} refers to information that can not be calculated using a deterministic approach, which allows a situation to be given context. Therefore acts as the source of information that need to be quantified to make a contextual-determination} into language that can be understood by a program. The second focuses on the reduction of bias when evaluating testing data\footnote{\textit{Testing data} refers to the data used to test a machine learning model that has been trained using \textit{training data}, by extension test data $\neq$ training data}. Even though these topics are not explored in this report, they are still worth mentioning as they will be my primary focus in the forthcoming months, as outlined in my Gantt Char, Appendix~\ref{fig:gantt}, and in Chapter~\ref{ch:aow.plan}.

Otherwise, the structure of this report is as follows. Chapter~\ref{ch:research} will focus on the body of research. More precisely Chapter~\ref{ch:philo} will focus on the philosophical reasoning implemented in my DM model, whilst in Chapter~\ref{ch:tech}, I will apply DM and learning techniques in order to develop an exemplar program. In Chapter~\ref{ch:prop} I propose the finalised philosophical framework and in and Chapter~\ref{ch:aow} I aim to emphasise the work done and strictly layout the plan going forward.

\section{Further Insight}\label{ch:intro.plan}
The goal is to assist a client in choosing the solution from a list of reasonable solution, that will yield the \textit{best}\footnote{The use of \textit{best} suggests a layer of subjectivity to our model - this is a common flaw that many decision-making robots succumb to. I must recognise that it is an inevitable flaw within my model as my choice to (or not to) include certain philosophies means my robot will be constrained to \textit{my} subjectivity, though I aim to generalise my solution so that the implication of my subjectivity in this manner is minimised} result. More specifically the aim is to develop a philosophical framework with applied DM and ML techniques in order to assist a cleints decision making process in an ethical dilemma, and further illustrate it's workings by use of program. The plan outlined by the Gantt Chart follows the development of the philosophical model and a corresponding program. My research methodology can be found in Chapter~\ref{ch:aow.plan}.

To briefly comment on the impact of COVID-19: there is little to fret about the construction of my model; whether philosophical or technological, is not constrained by any physical instruments or facilities, nor is/will there be any need for person-to-person contact.
%% ----------------------------------------------------------------
%% Conclusions.tex
%% ---------------------------------------------------------------- 
\chapter{Conclusions} \label{Chapter: Conclusions}
It works.

\appendix
%% ----------------------------------------------------------------
%% AppendixA.tex
%% ---------------------------------------------------------------- 
\chapter{Stuff} \label{Chapter:Stuff}
The following gets in the way of the text....

\backmatter
\bibliographystyle{ecs}
\bibliography{ECS}
%</testthesis|testminithesis|testprogress|testproject|testreport|testgdp>
%<*testgdpsummary>
\section*{Introduction}
\section*{Objectives}
\section*{Resources}
\section*{Constraints}
\section*{Approaching the task}
\section*{Team Organisation}
\section*{Important Results}
\section*{Conclusions}
\section*{Recommendations}
%</testgdpsummary>
%<*introduction>
\chapter{Introduction} \label{Chapter:Introduction}
%</introduction>
%<*testarticle>
\section{Introduction} \label{Section:Introduction}
%</testarticle>
%<*introduction|testarticle>
You probably found all the files from \cite{Gunn:2001:pdflatex}.
\tref{Table:tabex} illustrates the results of my work.
\begin{table}[!htb]
  \centering
  \begin{tabular}{cc}
  \toprule
  \textbf{Training Error} & \textbf{Testing Error}\\
  \midrule
  0 & $\infty$\\
  \bottomrule
  \end{tabular}
  \caption{The Results}
  \label{Table:tabex}
\end{table}

\fref{Figure:figex} shows why this is the case.
\begin{figure}[!htb]
  \centering
  \includegraphics[width=8cm]{figure}
  \caption{A colourful picture.}
  \label{Figure:figex}
\end{figure}

This page shows you a subfigure example in \fref{Figure:figsubex}.
\begin{figure}[!htb]
  \centering
  \subfigure[The left caption]{
    \includegraphics[width=4.2cm]{figure}
    \label{Figure:figsubex:left}
  }
  \subfigure[The right caption]{
    \includegraphics[width=4.2cm]{figure}
    \label{Figure:figsubex:right}
  }
  \caption{A doubly colourful picture.}
  \label{Figure:figsubex}
\end{figure}
%</introduction|testarticle>
%<*conclusions>
\chapter{Conclusions} \label{Chapter: Conclusions}
%</conclusions>
%<*testarticle>
%% ----------------------------------------------------------------
\section{Conclusions} \label{Section: Conclusions}
%</testarticle>
%<*conclusions|testarticle>
It works.
%</conclusions|testarticle>
%<*testarticle>
\acknowledgements{Thanks to no one.}
\backmatter
\bibliographystyle{ecs}
\bibliography{ECS}
\appendix
%</testarticle>
%<*appendix|testarticle>
%<*appendix>
\chapter{Stuff} \label{Chapter:Stuff}
%</appendix>
%<*testarticle>
%% ----------------------------------------------------------------
\section{Stuff} \label{Section:Stuff}
%</testarticle>
The following gets in the way of the text....
%</appendix|testarticle>
%<*testthesis|testminithesis|testprogress|testproject|testreport|testarticle|testgdp|testgdpsummary>
\end{document}
%% ----------------------------------------------------------------
%</testthesis|testminithesis|testprogress|testproject|testreport|testarticle|testgdp|testgdpsummary>
%    \end{macrocode}
%
% \subsection{Example Figure}
%
%    \begin{macrocode}
%<*figure>
%%BoundingBox: 0 150 600 650
%%EndComments
0.0 setlinewidth
/length 0.1 def
/width 0.02 def
/hsvcircle {
gsave
    /h 0.0 def
    0 4 360 {
    pop
    gsave
    0.5 0.0 translate
    newpath
    0.0 0.0 moveto
    length 0.0 lineto
    length width lineto
    0.0 width lineto
    closepath
    h 1.0 1.0 sethsbcolor
    fill
    grestore
    /h h 4 360 div add def
    4 rotate
    } for
grestore
} def
0.0 setlinewidth
0.0 setgray
300 400 translate
500 500 scale
30 rotate
1.0 0.7 scale
-30 rotate
hsvcircle
0.8 0.8 scale
hsvcircle
0.8 0.8 scale
hsvcircle
0.8 0.8 scale
hsvcircle
0.8 0.8 scale
hsvcircle
0.8 0.8 scale
hsvcircle
0.8 0.8 scale
hsvcircle
showpage
%</figure>
%    \end{macrocode}
%
% \subsection{Definitions File}
%
%    \begin{macrocode}
%<*definitions>
\newcommand{\BibTeX}{{\rm B\kern-.05em{\sc i\kern-.025em b}\kern-.08em T\kern-.1667em\lower.7ex\hbox{E}\kern-.125emX}}

%% People
\newcounter{address}
\setcounter{address}{1}
\renewcommand{\theaddress}{\textsuperscript{\fnsymbol{address}}}
\newcommand{\address}[1]{\refstepcounter{address}\theaddress#1\\}
\newcommand{\Name}[3]{\texorpdfstring{\href{mailto:#3}{#2}#1}{#2}\xspace}
\newcommand{\SteveRGunn}[1]{\Name{#1}{Steve R. Gunn}{S.R.Gunn@ecs.soton.ac.uk}}

%% Dingbats
\newcommand{\tick}{\ding{51}}
\newcommand{\cross}{\ding{55}}

%% Calculus
\newcommand{\pd}[2]{\ensuremath{\frac{\partial #1}{\partial #2}}\xspace}
\newcommand{\fd}[2]{\ensuremath{\frac{d #1}{d #2}}\xspace}
\newcommand{\dint}{\ensuremath{\int\!\!\!\int}\xspace}
\newcommand{\tint}{\ensuremath{\int\!\!\!\int\!\!\!\int}\xspace}

%% Math Sets
\newcommand{\Q}[1]{\ensuremath{\mathbb{#1}}\xspace}
\newcommand{\R}{\Q{R}}

%% Matrix, Vector
\newcommand{\V}[1]{\ensuremath{\boldsymbol{#1}}\xspace}
\newcommand{\M}[1]{\ensuremath{\boldsymbol{#1}}\xspace}
\newcommand{\0}{\V{0}}
\newcommand{\1}{\V{1}}
\newcommand{\I}{\M{I}}

%% Math Functions
\newcommand{\F}[1]{\ensuremath{\mathrm{#1}}\xspace}
\newcommand{\sgn}{\F{sgn}}
\newcommand{\tr}{\F{trace}}
\newcommand{\diag}{\F{diag}}

%% Math Names
\newcommand{\N}[1]{\ensuremath{\mathit{#1}}\xspace}

%% Data
\newcommand{\mc}[1]{\ensuremath{\mathcal{#1}}\xspace}
\newcommand{\Hyp}{\mc{H}}
\newcommand{\D}{\mc{D}}

%% Kernel
\newcommand{\K}{\M{K}}
\newcommand{\eins}{\texorpdfstring{\ensuremath{\epsilon}}{\textepsilon}-insensitive\xspace}
\newcommand{\e}{\ensuremath{\epsilon}\xspace}
\newcommand{\Bxi}{\ensuremath{\boldsymbol{\xi}}\xspace}
\newcommand{\Kanova}{\ensuremath{\mathit{K_{ANOVA}}}\xspace}
\newcommand{\Kspline}{\ensuremath{\mathit{K_{spline}}}\xspace}

%% Bayesian
\newcommand{\MP}{\ensuremath{\mathit{{\scriptscriptstyle \hspace{-1.5pt}M\hspace{-1.5pt}P}}}\xspace}
\newcommand{\ML}{\ensuremath{\mathit{{\scriptscriptstyle \hspace{-1.5pt}M\hspace{-1.5pt}L}}}\xspace}
\newcommand{\Qw}{\ensuremath{Q_{\w}(\w)}\xspace}
\newcommand{\Qa}{\ensuremath{Q_{\Ba}(\Ba)}\xspace}
\newcommand{\Qb}{\ensuremath{Q_{\beta}(\beta)}\xspace}
\newcommand{\wMPab}{\ensuremath{\w_{\MP|\bar {\Ba},\bar \beta}}\xspace}
\newcommand{\wMP}{\ensuremath{\w_{\MP}}\xspace}
\newcommand{\yMP}{\ensuremath{y_{\MP}}\xspace}
\newcommand{\BaMP}{\ensuremath{\Ba_{\hspace{1pt}\MP}}\xspace}
\newcommand{\aMP}{\ensuremath{\alpha_{\hspace{1pt}\MP}}\xspace}
\newcommand{\bMP}{\ensuremath{\beta_{\hspace{1pt}\MP}}\xspace}
\newcommand{\Sab}{\ensuremath{\M{\Sigma}_{\bar \Ba,\bar \beta}}\xspace}
\newcommand{\Ba}{\ensuremath{\boldsymbol{\alpha}}\xspace}
\newcommand{\Bb}{\ensuremath{\boldsymbol{\beta}}\xspace}
\newcommand{\Bm}{\ensuremath{\boldsymbol{\mu}}\xspace}
\newcommand{\BL}{\ensuremath{\boldsymbol{\Lambda}}\xspace}
\newcommand{\BPhi}{\ensuremath{\boldsymbol{\Phi}}\xspace}
\newcommand{\SMP}{\ensuremath{\M{\Sigma}_{\MP}}\xspace}

\newcommand{\Pa}{\ensuremath{P(\alpha|\mathcal{H})}\xspace}
\newcommand{\Pb}{\ensuremath{P(\beta|\mathcal{H})}\xspace}
\newcommand{\Pab}{\ensuremath{P(\alpha,\beta|\mathcal{H})}\xspace}
\newcommand{\Pw}{\ensuremath{P(\w|\mathcal{H})}\xspace}
\newcommand{\PD}{\ensuremath{P(\D|\mathcal{H})}\xspace}
\newcommand{\PwIa}{\ensuremath{P(\w|\alpha,\mathcal{H})}\xspace}
\newcommand{\PDIwb}{\ensuremath{P(\D|\w,\beta,\mathcal{H})}\xspace}
\newcommand{\PDwab}{\ensuremath{P(\D,\w,\alpha,\beta|\mathcal{H})}\xspace}
\newcommand{\PDIw}{\ensuremath{P(\D|\w,\mathcal{H})}\xspace}
\newcommand{\PwID}{\ensuremath{P(\w|\D,\mathcal{H})}\xspace}
\newcommand{\PwabID}{\ensuremath{P(\w,\alpha,\beta|\D,\mathcal{H})}\xspace}

\newcommand{\PanH}{\ensuremath{P(\alpha)}\xspace}
\newcommand{\PbnH}{\ensuremath{P(\beta)}\xspace}
\newcommand{\PabnH}{\ensuremath{P(\alpha,\beta)}\xspace}
\newcommand{\PwnH}{\ensuremath{P(\w)}\xspace}
\newcommand{\PDnH}{\ensuremath{P(\D)}\xspace}
\newcommand{\PwIanH}{\ensuremath{P(\w|\alpha)}\xspace}
\newcommand{\PDIwbnH}{\ensuremath{P(\D|\w,\beta)}\xspace}
\newcommand{\PDwabnH}{\ensuremath{P(\D,\w,\Ba,\beta)}\xspace}
\newcommand{\PDIwnH}{\ensuremath{P(\D|\w)}\xspace}
\newcommand{\PwIDnH}{\ensuremath{P(\w|\D)}\xspace}
\newcommand{\PwabIDnH}{\ensuremath{P(\w,\alpha,\beta|\D)}\xspace}

\newcommand{\PDwBab}{\ensuremath{P(\D,\w,\Ba,\beta|\mathcal{H})}\xspace}
\newcommand{\PwIBa}{\ensuremath{P(\w|\Ba,\mathcal{H})}\xspace}
\newcommand{\PBab}{\ensuremath{P(\Ba,\beta|\mathcal{H})}\xspace}
\newcommand{\PwBabID}{\ensuremath{P(\w,\Ba,\beta|\D,\mathcal{H})}\xspace}

\newcommand{\PBanH}{\ensuremath{P(\Ba)}\xspace}
\newcommand{\PwIBanH}{\ensuremath{P(\w|\Ba)}\xspace}

%% Snakes
\newcommand{\Esnake}{\ensuremath{\mathit{E_{snake}}}\xspace}
\newcommand{\Eimage}{\ensuremath{\mathit{E_{image}}}\xspace}
\newcommand{\Econt}{\ensuremath{\mathit{E_{cont}}}\xspace}
\newcommand{\Ecurv}{\ensuremath{\mathit{E_{curv}}}\xspace}
\newcommand{\Eint}{\ensuremath{\mathit{E_{int}}}\xspace}
\newcommand{\Eext}{\ensuremath{\mathit{E_{ext}}}\xspace}
\newcommand{\Eterm}{\ensuremath{\mathit{E_{term}}}\xspace}
\newcommand{\Eline}{\ensuremath{\mathit{E_{line}}}\xspace}
\newcommand{\Eedge}{\ensuremath{\mathit{E_{edge}}}\xspace}
\newcommand{\Econ}{\ensuremath{\mathit{E_{con}}}\xspace}
\newcommand{\Eangle}{\ensuremath{\mathit{E_{angle}}}\xspace}
\newcommand{\Elshape}{\ensuremath{\mathit{E_{lshape}}}\xspace}
\newcommand{\Eedgedir}{\ensuremath{\mathit{E_{edgedir}}}\xspace}
\newcommand{\Emodel}{\ensuremath{\mathit{E_{model}}}\xspace}
\newcommand{\wte}{\ensuremath{\mathit{w_{term}}}\xspace}
\newcommand{\wli}{\ensuremath{\mathit{w_{line}}}\xspace}
\newcommand{\wed}{\ensuremath{\mathit{w_{edge}}}\xspace}
\newcommand{\wco}{\ensuremath{\mathit{w_{con}}}\xspace}


%% Environments
\newcounter{alg}
\newenvironment{algorithm}[1]
{
    \stepcounter{alg}
    \begin{table}[htb]
    \centering
    \begin{tabular}[t]{ll}
    \hline&\\
    \multicolumn{2}{l}{\bf Algorithm \arabic{alg}: #1}\\&\\
} {
    &\\
    \hline
    \end{tabular}
    \end{table}
}
%</definitions>
%    \end{macrocode}
%
% \subsection{References File}
%
%    \begin{macrocode}
%<*references>
@MISC{Gunn:2001:pdflatex,
  author =       {S.R. Gunn},
  title =        {PDFLaTeX Instructions},
  year =         {2001},
  url =          {http://www.ecs.soton.ac.uk/~srg/softwaretools/document/}
}
%</references>
%    \end{macrocode}
%
% \subsection{Bibliography Style File}
%
%    \begin{macrocode}
%<*bst>
%% File: `ecs.bst'
%% A modification of `plainnat.bst' for use with natbib package
%%
%% Copyright 1993-1999 Patrick W Daly
%% Max-Planck-Institut f\"ur Aeronomie
%% Max-Planck-Str. 2
%% D-37191 Katlenburg-Lindau
%% Germany
%% E-mail: daly@linmpi.mpg.de
%%
%% This program can be redistributed and/or modified under the terms
%% of the LaTeX Project Public License Distributed from CTAN
%% archives in directory macros/latex/base/lppl.txt; either
%% version 1 of the License, or any later version.
%%
%% Includes field URL for Internet addresses
%%
ENTRY
  { address
    author
    booktitle
    chapter
    edition
    editor
    howpublished
    institution
    isbn
    issn
    journal
    key
    month
    note
    number
    organization
    pages
    publisher
    school
    series
    title
    type
    url
    volume
    year
  }
  {}
  { label extra.label sort.label short.list }

INTEGERS { output.state before.all mid.sentence after.sentence after.block }

FUNCTION {init.state.consts}
{ #0 'before.all :=
  #1 'mid.sentence :=
  #2 'after.sentence :=
  #3 'after.block :=
}

STRINGS { s t }

FUNCTION {output.nonnull}
{ 's :=
  output.state mid.sentence =
    { ", " * write$ }
    { output.state after.block =
        { add.period$ write$
          newline$
          "\newblock " write$
        }
        { output.state before.all =
            'write$
            { add.period$ " " * write$ }
          if$
        }
      if$
      mid.sentence 'output.state :=
    }
  if$
  s
}

FUNCTION {output}
{ duplicate$ empty$
    'pop$
    'output.nonnull
  if$
}

FUNCTION {output.check}
{ 't :=
  duplicate$ empty$
    { pop$ "empty " t * " in " * cite$ * warning$ }
    'output.nonnull
  if$
}

FUNCTION {fin.entry}
{ add.period$
  write$
  newline$
}

FUNCTION {new.block}
{ output.state before.all =
    'skip$
    { after.block 'output.state := }
  if$
}

FUNCTION {new.sentence}
{ output.state after.block =
    'skip$
    { output.state before.all =
        'skip$
        { after.sentence 'output.state := }
      if$
    }
  if$
}

FUNCTION {not}
{   { #0 }
    { #1 }
  if$
}

FUNCTION {and}
{   'skip$
    { pop$ #0 }
  if$
}

FUNCTION {or}
{   { pop$ #1 }
    'skip$
  if$
}

FUNCTION {new.block.checka}
{ empty$
    'skip$
    'new.block
  if$
}

FUNCTION {new.block.checkb}
{ empty$
  swap$ empty$
  and
    'skip$
    'new.block
  if$
}

FUNCTION {new.sentence.checka}
{ empty$
    'skip$
    'new.sentence
  if$
}

FUNCTION {new.sentence.checkb}
{ empty$
  swap$ empty$
  and
    'skip$
    'new.sentence
  if$
}

FUNCTION {field.or.null}
{ duplicate$ empty$
    { pop$ "" }
    'skip$
  if$
}

FUNCTION {emphasize}
{ duplicate$ empty$
    { pop$ "" }
    { "{\em " swap$ * "}" * }
  if$
}

INTEGERS { nameptr namesleft numnames }

FUNCTION {format.names}
{ 's :=
  #1 'nameptr :=
  s num.names$ 'numnames :=
  numnames 'namesleft :=
    { namesleft #0 > }
    { s nameptr "{ff~}{vv~}{ll}{, jj}" format.name$ 't :=
      nameptr #1 >
        { namesleft #1 >
            { ", " * t * }
            { numnames #2 >
                { "," * }
                'skip$
              if$
              t "others" =
                { " et~al." * }
                { " and " * t * }
              if$
            }
          if$
        }
        't
      if$
      nameptr #1 + 'nameptr :=
      namesleft #1 - 'namesleft :=
    }
  while$
}

FUNCTION {format.key}
{ empty$
    { key field.or.null }
    { "" }
  if$
}

FUNCTION {format.authors}
{ author empty$
    { "" }
    { author format.names }
  if$
}

FUNCTION {format.editors}
{ editor empty$
    { "" }
    { editor format.names
      editor num.names$ #1 >
        { ", editors" * }
        { ", editor" * }
      if$
    }
  if$
}

FUNCTION {format.isbn}
{ isbn empty$
    { "" }
    { new.block "ISBN " isbn * }
  if$
}

FUNCTION {format.issn}
{ issn empty$
    { "" }
    { new.block "ISSN " issn * }
  if$
}

%FUNCTION {format.url}
%{ url empty$
%    { "" }
%    { new.block "URL \url{" url * "}" * }
%  if$
%}
%
%FUNCTION {format.title}
%{ title empty$
%    { "" }
%    { title "t" change.case$ }
%  if$
%}
FUNCTION {format.url}
{ ""
}

FUNCTION {format.title}
{ title empty$
    { "" }
    { url empty$
        { title "t" change.case$ }
        { "\href{" url * "}{" * title "t" change.case$ * "}" * }
      if$
    }
    if$
}


FUNCTION {format.full.names}
{'s :=
  #1 'nameptr :=
  s num.names$ 'numnames :=
  numnames 'namesleft :=
    { namesleft #0 > }
    { s nameptr
      "{vv~}{ll}" format.name$ 't :=
      nameptr #1 >
        {
          namesleft #1 >
            { ", " * t * }
            {
              numnames #2 >
                { "," * }
                'skip$
              if$
              t "others" =
                { " et~al." * }
                { " and " * t * }
              if$
            }
          if$
        }
        't
      if$
      nameptr #1 + 'nameptr :=
      namesleft #1 - 'namesleft :=
    }
  while$
}

FUNCTION {author.editor.full}
{ author empty$
    { editor empty$
        { "" }
        { editor format.full.names }
      if$
    }
    { author format.full.names }
  if$
}

FUNCTION {author.full}
{ author empty$
    { "" }
    { author format.full.names }
  if$
}

FUNCTION {editor.full}
{ editor empty$
    { "" }
    { editor format.full.names }
  if$
}

FUNCTION {make.full.names}
{ type$ "book" =
  type$ "inbook" =
  or
    'author.editor.full
    { type$ "proceedings" =
        'editor.full
        'author.full
      if$
    }
  if$
}

FUNCTION {output.bibitem}
{ newline$
  "\bibitem[" write$
  label write$
  ")" make.full.names duplicate$ short.list =
     { pop$ }
     { * }
   if$
  "]{" * write$
  cite$ write$
  "}" write$
  newline$
  ""
  before.all 'output.state :=
}

FUNCTION {n.dashify}
{ 't :=
  ""
    { t empty$ not }
    { t #1 #1 substring$ "-" =
        { t #1 #2 substring$ "--" = not
            { "--" *
              t #2 global.max$ substring$ 't :=
            }
            {   { t #1 #1 substring$ "-" = }
                { "-" *
                  t #2 global.max$ substring$ 't :=
                }
              while$
            }
          if$
        }
        { t #1 #1 substring$ *
          t #2 global.max$ substring$ 't :=
        }
      if$
    }
  while$
}

FUNCTION {format.date}
{ year duplicate$ empty$
    { "empty year in " cite$ * warning$
       pop$ "" }
    'skip$
  if$
  month empty$
    'skip$
    { month
      " " * swap$ *
    }
  if$
  extra.label *
}

FUNCTION {format.btitle}
{ title emphasize
}

FUNCTION {tie.or.space.connect}
{ duplicate$ text.length$ #3 <
    { "~" }
    { " " }
  if$
  swap$ * *
}

FUNCTION {either.or.check}
{ empty$
    'pop$
    { "can't use both " swap$ * " fields in " * cite$ * warning$ }
  if$
}

FUNCTION {format.bvolume}
{ volume empty$
    { "" }
    { "volume" volume tie.or.space.connect
      series empty$
        'skip$
        { " of " * series emphasize * }
      if$
      "volume and number" number either.or.check
    }
  if$
}

FUNCTION {format.number.series}
{ volume empty$
    { number empty$
        { series field.or.null }
        { output.state mid.sentence =
            { "number" }
            { "Number" }
          if$
          number tie.or.space.connect
          series empty$
            { "there's a number but no series in " cite$ * warning$ }
            { " in " * series * }
          if$
        }
      if$
    }
    { "" }
  if$
}

FUNCTION {format.edition}
{ edition empty$
    { "" }
    { output.state mid.sentence =
        { edition "l" change.case$ " edition" * }
        { edition "t" change.case$ " edition" * }
      if$
    }
  if$
}

INTEGERS { multiresult }

FUNCTION {multi.page.check}
{ 't :=
  #0 'multiresult :=
    { multiresult not
      t empty$ not
      and
    }
    { t #1 #1 substring$
      duplicate$ "-" =
      swap$ duplicate$ "," =
      swap$ "+" =
      or or
        { #1 'multiresult := }
        { t #2 global.max$ substring$ 't := }
      if$
    }
  while$
  multiresult
}

FUNCTION {format.pages}
{ pages empty$
    { "" }
    { pages multi.page.check
        { "pages" pages n.dashify tie.or.space.connect }
        { "page" pages tie.or.space.connect }
      if$
    }
  if$
}

FUNCTION {format.vol.num.pages}
{ volume field.or.null
  number empty$
    'skip$
    { "\penalty0 (" number * ")" * *
      volume empty$
        { "there's a number but no volume in " cite$ * warning$ }
        'skip$
      if$
    }
  if$
  pages empty$
    'skip$
    { duplicate$ empty$
        { pop$ format.pages }
        { ":\penalty0 " * pages n.dashify * }
      if$
    }
  if$
}

FUNCTION {format.chapter.pages}
{ chapter empty$
    'format.pages
    { type empty$
        { "chapter" }
        { type "l" change.case$ }
      if$
      chapter tie.or.space.connect
      pages empty$
        'skip$
        { ", " * format.pages * }
      if$
    }
  if$
}

FUNCTION {format.in.ed.booktitle}
{ booktitle empty$
    { "" }
    { editor empty$
        { "In " booktitle emphasize * }
        { "In " format.editors * ", " * booktitle emphasize * }
      if$
    }
  if$
}

FUNCTION {empty.misc.check}
{ author empty$ title empty$ howpublished empty$
  month empty$ year empty$ note empty$
  and and and and and
  key empty$ not and
    { "all relevant fields are empty in " cite$ * warning$ }
    'skip$
  if$
}

FUNCTION {format.thesis.type}
{ type empty$
    'skip$
    { pop$
      type "t" change.case$
    }
  if$
}

FUNCTION {format.tr.number}
{ type empty$
    { "Technical Report" }
    'type
  if$
  number empty$
    { "t" change.case$ }
    { number tie.or.space.connect }
  if$
}

FUNCTION {format.article.crossref}
{ key empty$
    { journal empty$
        { "need key or journal for " cite$ * " to crossref " * crossref *
          warning$
          ""
        }
        { "In {\em " journal * "\/}" * }
      if$
    }
    { "In " key * }
  if$
  " \citep{" * crossref * "}" *
}

FUNCTION {format.book.crossref}
{ volume empty$
    { "empty volume in " cite$ * "'s crossref of " * crossref * warning$
      "In "
    }
    { "Volume" volume tie.or.space.connect
      " of " *
    }
  if$
  editor empty$
  editor field.or.null author field.or.null =
  or
    { key empty$
        { series empty$
            { "need editor, key, or series for " cite$ * " to crossref " *
              crossref * warning$
              "" *
            }
            { "{\em " * series * "\/}" * }
          if$
        }
        { key * }
      if$
    }
    'skip$
  if$
  ", \citet{" * crossref * "}" *
}

FUNCTION {format.incoll.inproc.crossref}
{ editor empty$
  editor field.or.null author field.or.null =
  or
    { key empty$
        { booktitle empty$
            { "need editor, key, or booktitle for " cite$ * " to crossref " *
              crossref * warning$
              ""
            }
            { "In {\em " booktitle * "\/}" * }
          if$
        }
        { "In " key * }
      if$
    }
    { "In " }
  if$
  " \citet{" * crossref * "}" *
}

FUNCTION {article}
{ output.bibitem
  format.authors "author" output.check
  author format.key output
  new.block
  format.title "title" output.check
  new.block
  crossref missing$
    { journal emphasize "journal" output.check
      format.vol.num.pages output
      format.date "year" output.check
    }
    { format.article.crossref output.nonnull
      format.pages output
    }
  if$
  format.issn output
  format.url output
  new.block
  note output
  fin.entry
}

FUNCTION {book}
{ output.bibitem
  author empty$
    { format.editors "author and editor" output.check
      editor format.key output
    }
    { format.authors output.nonnull
      crossref missing$
        { "author and editor" editor either.or.check }
        'skip$
      if$
    }
  if$
  new.block
  format.btitle "title" output.check
  crossref missing$
    { format.bvolume output
      new.block
      format.number.series output
      new.sentence
      publisher "publisher" output.check
      address output
    }
    { new.block
      format.book.crossref output.nonnull
    }
  if$
  format.edition output
  format.date "year" output.check
  format.isbn output
  format.url output
  new.block
  note output
  fin.entry
}

FUNCTION {booklet}
{ output.bibitem
  format.authors output
  author format.key output
  new.block
  format.title "title" output.check
  howpublished address new.block.checkb
  howpublished output
  address output
  format.date output
  format.isbn output
  format.url output
  new.block
  note output
  fin.entry
}

FUNCTION {inbook}
{ output.bibitem
  author empty$
    { format.editors "author and editor" output.check
      editor format.key output
    }
    { format.authors output.nonnull
      crossref missing$
        { "author and editor" editor either.or.check }
        'skip$
      if$
    }
  if$
  new.block
  format.btitle "title" output.check
  crossref missing$
    { format.bvolume output
      format.chapter.pages "chapter and pages" output.check
      new.block
      format.number.series output
      new.sentence
      publisher "publisher" output.check
      address output
    }
    { format.chapter.pages "chapter and pages" output.check
      new.block
      format.book.crossref output.nonnull
    }
  if$
  format.edition output
  format.date "year" output.check
  format.isbn output
  format.url output
  new.block
  note output
  fin.entry
}

FUNCTION {incollection}
{ output.bibitem
  format.authors "author" output.check
  author format.key output
  new.block
  format.title "title" output.check
  new.block
  crossref missing$
    { format.in.ed.booktitle "booktitle" output.check
      format.bvolume output
      format.number.series output
      format.chapter.pages output
      new.sentence
      publisher "publisher" output.check
      address output
      format.edition output
      format.date "year" output.check
    }
    { format.incoll.inproc.crossref output.nonnull
      format.chapter.pages output
    }
  if$
  format.isbn output
  format.url output
  new.block
  note output
  fin.entry
}

FUNCTION {inproceedings}
{ output.bibitem
  format.authors "author" output.check
  author format.key output
  new.block
  format.title "title" output.check
  new.block
  crossref missing$
    { format.in.ed.booktitle "booktitle" output.check
      format.bvolume output
      format.number.series output
      format.pages output
      address empty$
        { organization publisher new.sentence.checkb
          organization output
          publisher output
          format.date "year" output.check
        }
        { address output.nonnull
          format.date "year" output.check
          new.sentence
          organization output
          publisher output
        }
      if$
    }
    { format.incoll.inproc.crossref output.nonnull
      format.pages output
    }
  if$
  format.isbn output
  format.url output
  new.block
  note output
  fin.entry
}

FUNCTION {conference} { inproceedings }

FUNCTION {manual}
{ output.bibitem
  format.authors output
  author format.key output
  new.block
  format.btitle "title" output.check
  organization address new.block.checkb
  organization output
  address output
  format.edition output
  format.date output
  format.url output
  new.block
  note output
  fin.entry
}

FUNCTION {mastersthesis}
{ output.bibitem
  format.authors "author" output.check
  author format.key output
  new.block
  format.title "title" output.check
  new.block
  "Master's thesis" format.thesis.type output.nonnull
  school "school" output.check
  address output
  format.date "year" output.check
  format.url output
  new.block
  note output
  fin.entry
}

FUNCTION {misc}
{ output.bibitem
  format.authors output
  author format.key output
  title howpublished new.block.checkb
  format.title output
  howpublished new.block.checka
  howpublished output
  format.date output
  format.issn output
  format.url output
  new.block
  note output
  fin.entry
  empty.misc.check
}

FUNCTION {phdthesis}
{ output.bibitem
  format.authors "author" output.check
  author format.key output
  new.block
  format.btitle "title" output.check
  new.block
  "PhD thesis" format.thesis.type output.nonnull
  school "school" output.check
  address output
  format.date "year" output.check
  format.url output
  new.block
  note output
  fin.entry
}

FUNCTION {proceedings}
{ output.bibitem
  format.editors output
  editor format.key output
  new.block
  format.btitle "title" output.check
  format.bvolume output
  format.number.series output
  address output
  format.date "year" output.check
  new.sentence
  organization output
  publisher output
  format.isbn output
  format.url output
  new.block
  note output
  fin.entry
}

FUNCTION {techreport}
{ output.bibitem
  format.authors "author" output.check
  author format.key output
  new.block
  format.title "title" output.check
  new.block
  format.tr.number output.nonnull
  institution "institution" output.check
  address output
  format.date "year" output.check
  format.url output
  new.block
  note output
  fin.entry
}

FUNCTION {unpublished}
{ output.bibitem
  format.authors "author" output.check
  author format.key output
  new.block
  format.title "title" output.check
  format.url output
  new.block
  note "note" output.check
  format.date output
  fin.entry
}

FUNCTION {default.type} { misc }


MACRO {jan} {"January"}

MACRO {feb} {"February"}

MACRO {mar} {"March"}

MACRO {apr} {"April"}

MACRO {may} {"May"}

MACRO {jun} {"June"}

MACRO {jul} {"July"}

MACRO {aug} {"August"}

MACRO {sep} {"September"}

MACRO {oct} {"October"}

MACRO {nov} {"November"}

MACRO {dec} {"December"}



MACRO {acmcs} {"ACM Computing Surveys"}

MACRO {acta} {"Acta Informatica"}

MACRO {cacm} {"Communications of the ACM"}

MACRO {ibmjrd} {"IBM Journal of Research and Development"}

MACRO {ibmsj} {"IBM Systems Journal"}

MACRO {ieeese} {"IEEE Transactions on Software Engineering"}

MACRO {ieeetc} {"IEEE Transactions on Computers"}

MACRO {ieeetcad}
 {"IEEE Transactions on Computer-Aided Design of Integrated Circuits"}

MACRO {ipl} {"Information Processing Letters"}

MACRO {jacm} {"Journal of the ACM"}

MACRO {jcss} {"Journal of Computer and System Sciences"}

MACRO {scp} {"Science of Computer Programming"}

MACRO {sicomp} {"SIAM Journal on Computing"}

MACRO {tocs} {"ACM Transactions on Computer Systems"}

MACRO {tods} {"ACM Transactions on Database Systems"}

MACRO {tog} {"ACM Transactions on Graphics"}

MACRO {toms} {"ACM Transactions on Mathematical Software"}

MACRO {toois} {"ACM Transactions on Office Information Systems"}

MACRO {toplas} {"ACM Transactions on Programming Languages and Systems"}

MACRO {tcs} {"Theoretical Computer Science"}


READ

FUNCTION {sortify}
{ purify$
  "l" change.case$
}

INTEGERS { len }

FUNCTION {chop.word}
{ 's :=
  'len :=
  s #1 len substring$ =
    { s len #1 + global.max$ substring$ }
    's
  if$
}

FUNCTION {format.lab.names}
{ 's :=
  s #1 "{vv~}{ll}" format.name$
  s num.names$ duplicate$
  #2 >
    { pop$ " et~al." * }
    { #2 <
        'skip$
        { s #2 "{ff }{vv }{ll}{ jj}" format.name$ "others" =
            { " et~al." * }
            { " and " * s #2 "{vv~}{ll}" format.name$ * }
          if$
        }
      if$
    }
  if$
}

FUNCTION {author.key.label}
{ author empty$
    { key empty$
        { cite$ #1 #3 substring$ }
        'key
      if$
    }
    { author format.lab.names }
  if$
}

FUNCTION {author.editor.key.label}
{ author empty$
    { editor empty$
        { key empty$
            { cite$ #1 #3 substring$ }
            'key
          if$
        }
        { editor format.lab.names }
      if$
    }
    { author format.lab.names }
  if$
}

FUNCTION {author.key.organization.label}
{ author empty$
    { key empty$
        { organization empty$
            { cite$ #1 #3 substring$ }
            { "The " #4 organization chop.word #3 text.prefix$ }
          if$
        }
        'key
      if$
    }
    { author format.lab.names }
  if$
}

FUNCTION {editor.key.organization.label}
{ editor empty$
    { key empty$
        { organization empty$
            { cite$ #1 #3 substring$ }
            { "The " #4 organization chop.word #3 text.prefix$ }
          if$
        }
        'key
      if$
    }
    { editor format.lab.names }
  if$
}

FUNCTION {calc.short.authors}
{ type$ "book" =
  type$ "inbook" =
  or
    'author.editor.key.label
    { type$ "proceedings" =
        'editor.key.organization.label
        { type$ "manual" =
            'author.key.organization.label
            'author.key.label
          if$
        }
      if$
    }
  if$
  'short.list :=
}

FUNCTION {calc.label}
{ calc.short.authors
  short.list
  "("
  *
  year duplicate$ empty$
  short.list key field.or.null = or
     { pop$ "" }
     'skip$
  if$
  *
  'label :=
}

FUNCTION {sort.format.names}
{ 's :=
  #1 'nameptr :=
  ""
  s num.names$ 'numnames :=
  numnames 'namesleft :=
    { namesleft #0 > }
    { nameptr #1 >
        { "   " * }
        'skip$
      if$
      s nameptr "{vv{ } }{ll{ }}{  ff{ }}{  jj{ }}" format.name$ 't :=
      nameptr numnames = t "others" = and
        { "et al" * }
        { t sortify * }
      if$
      nameptr #1 + 'nameptr :=
      namesleft #1 - 'namesleft :=
    }
  while$
}

FUNCTION {sort.format.title}
{ 't :=
  "A " #2
    "An " #3
      "The " #4 t chop.word
    chop.word
  chop.word
  sortify
  #1 global.max$ substring$
}

FUNCTION {author.sort}
{ author empty$
    { key empty$
        { "to sort, need author or key in " cite$ * warning$
          ""
        }
        { key sortify }
      if$
    }
    { author sort.format.names }
  if$
}

FUNCTION {author.editor.sort}
{ author empty$
    { editor empty$
        { key empty$
            { "to sort, need author, editor, or key in " cite$ * warning$
              ""
            }
            { key sortify }
          if$
        }
        { editor sort.format.names }
      if$
    }
    { author sort.format.names }
  if$
}

FUNCTION {author.organization.sort}
{ author empty$
    { organization empty$
        { key empty$
            { "to sort, need author, organization, or key in " cite$ * warning$
              ""
            }
            { key sortify }
          if$
        }
        { "The " #4 organization chop.word sortify }
      if$
    }
    { author sort.format.names }
  if$
}

FUNCTION {editor.organization.sort}
{ editor empty$
    { organization empty$
        { key empty$
            { "to sort, need editor, organization, or key in " cite$ * warning$
              ""
            }
            { key sortify }
          if$
        }
        { "The " #4 organization chop.word sortify }
      if$
    }
    { editor sort.format.names }
  if$
}

FUNCTION {presort}
{ calc.label
  label sortify
  "    "
  *
  type$ "book" =
  type$ "inbook" =
  or
    'author.editor.sort
    { type$ "proceedings" =
        'editor.organization.sort
        { type$ "manual" =
            'author.organization.sort
            'author.sort
          if$
        }
      if$
    }
  if$
  "    "
  *
  year field.or.null sortify
  *
  "    "
  *
  title field.or.null
  sort.format.title
  *
  #1 entry.max$ substring$
  'sort.label :=
  sort.label *
  #1 entry.max$ substring$
  'sort.key$ :=
}

ITERATE {presort}

SORT

STRINGS { longest.label last.label next.extra }

INTEGERS { longest.label.width last.extra.num number.label }

FUNCTION {initialize.longest.label}
{ "" 'longest.label :=
  #0 int.to.chr$ 'last.label :=
  "" 'next.extra :=
  #0 'longest.label.width :=
  #0 'last.extra.num :=
  #0 'number.label :=
}

FUNCTION {forward.pass}
{ last.label label =
    { last.extra.num #1 + 'last.extra.num :=
      last.extra.num int.to.chr$ 'extra.label :=
    }
    { "a" chr.to.int$ 'last.extra.num :=
      "" 'extra.label :=
      label 'last.label :=
    }
  if$
  number.label #1 + 'number.label :=
}

FUNCTION {reverse.pass}
{ next.extra "b" =
    { "a" 'extra.label := }
    'skip$
  if$
  extra.label 'next.extra :=
  extra.label
  duplicate$ empty$
    'skip$
    { "{\natexlab{" swap$ * "}}" * }
  if$
  'extra.label :=
  label extra.label * 'label :=
}

EXECUTE {initialize.longest.label}

ITERATE {forward.pass}

REVERSE {reverse.pass}

FUNCTION {bib.sort.order}
{ sort.label  'sort.key$ :=
}

ITERATE {bib.sort.order}

SORT

FUNCTION {begin.bib}
{   preamble$ empty$
    'skip$
    { preamble$ write$ newline$ }
  if$
  "\begin{thebibliography}{" number.label int.to.str$ * "}" *
  write$ newline$
  "\expandafter\ifx\csname natexlab\endcsname\relax\def\natexlab#1{#1}\fi"
  write$ newline$
  "\expandafter\ifx\csname url\endcsname\relax" write$ newline$
  "  \def\url#1{{\tt #1}}\fi" write$ newline$
}

EXECUTE {begin.bib}

EXECUTE {init.state.consts}

ITERATE {call.type$}

FUNCTION {end.bib}
{ newline$
  "\end{thebibliography}" write$ newline$
}

EXECUTE {end.bib}
%</bst>
%    \end{macrocode}
%
% \Finale
%
\endinput
